\chapter{Creando la informaci\'on}
\label{sec:creandoLaInformacion}
La creaci\'on de la informaci\'on se logra mediante pantallas de creaci\'on
independientes para cada elemento las cuales pueden ser accedidas por dos
m\'etodos principales:

\begin{description}
  \item[M\'etodo 1:]Por medio del men\'u \blackberry en la pantalla inicial,
  all\'i se selecciona \textit{Nuevo} (Fig.\ref{fig:nuevoPantallaInicial}) y
elige el elemento que desea crear en la lista
presentada (Fig.\ref{fig:listadoCrear})

\insertImage{PantallaInicial/MenuPrincipal-Nuevo}{Men\'u \emph{Nuevo} en la
pantalla inicial}{nuevoPantallaInicial}
\insertImage{Varios/ListaNuevo}{Nuevos elementos}{listadoCrear}
  \item[M\'etodo 2:]Por medio del men\'u \blackberry en la pantalla inicial,
  all\'i selecciona \textit{Listado} (Fig.\ref{fig:listadoPantallaInicial}) y
elige el listado del tipo que desea (Fig. \ref{fig:listadoListados}),
  despu\'es despliega el men\'u \blackberry y selecciona la opci\'on
\emph{Nuevo} (Fig. \ref{fig:nuevoElemento})
\insertImage{PantallaInicial/MenuPrincipal-Listado}{Men\'u \emph{Listado} en la
pantalla inicial}{listadoPantallaInicial}
\insertImage{Varios/ListaListado}{Listados de elementos}{listadoListados}
\insertImage{Listados/Demandante/Nuevo}{Nuevo elemento desde
listado}{nuevoElemento}
\end{description}

A continuaci\'on se describe el procedimiento para cada elemento usado por la
aplicaci\'on.

\section{Nuevo demandante}
\label{sec:nuevoDemandante}
Dir\'ijase a la pantalla inicial, despliegue el men\'u \blackberry y seleccione
\emph{Nuevo}, en este punto se le muestra un listado de los elementos posibles y
all\'i elija la opci\'on \emph{Demandante}.

Se presentar\'a una pantalla en la que podr\'a introducir toda la informaci\'on
que puede contener un demandante
\footnote{El campo Nombre se considera obligatorio y el campo Tel\'efono se
considera importante pero opcional},
cuando termine de ingresarla despliegue el men\'u \blackberry y seleccione
\emph{Guardar}.




\section{Nuevo demandado}
\label{sec:nuevoDemandado}
Dir\'ijase a la pantalla inicial, despliegue el men\'u \blackberry y seleccione
\emph{Nuevo}, en este punto se le muestra un listado de los elementos posibles y
all\'i elija la opci\'on \emph{Demandado}.

Se presentar\'a una pantalla en la que podr\'a introducir toda la informaci\'on
que puede contener un demandado
\footnote{El campo Nombre se considera obligatorio y el campo Tel\'efono se
considera importante pero opcional},
cuando termine de ingresarla despliegue el men\'u \blackberry y seleccione
\emph{Guardar}.
\section{Nuevo juzgado}
\label{sec:nuevoJuzgado}
Dir\'ijase a la pantalla inicial, despliegue el men\'u \blackberry y seleccione
\emph{Nuevo}, en este punto se le muestra un listado de los elementos posibles y
all\'i elija la opci\'on \emph{Juzgado}.

Se presentar\'a una pantalla en la que podr\'a introducir toda la informaci\'on
que puede contener un juzgado
\footnote{El campo Nombre se considera obligatorio y el campo Tel\'efono se
considera importante pero opcional},
cuando termine de ingresarla despliegue el men\'u \blackberry y seleccione
\emph{Guardar}.
\section{Nuevo campo personalizado}
\label{sec:nuevoCampo}
Dir\'ijase a la pantalla inicial, despliegue el men\'u \blackberry y seleccione
\emph{Nuevo}, en este punto se le muestra un listado de los elementos posibles y
all\'i elija la opci\'on \emph{Campo personalizado}.

Se presentar\'a una pantalla en la que podr\'a introducir el nombre del nuevo
campo personalizado
\footnote{El campo Nombre se considera obligatorio},
as\'i como tres
alternativas:
\begin{description}
\item[Obligatorio:]Es un campo de selecci\'on, que le permite determinar si el
campo que est\'a siendo creado, cuando se agregue a un proceso siempre
debe contener informaci\'on.
\item[Longitud m\'axima:]Es un campo para insertar n\'umeros, este tiene la
funci\'on de indicar que el campo que est\'a siendo creado, cuando se agregue a un proceso siempre debe tener informaci\'on de una longitud
m\'axima del valor introducido.
\footnote{Al ingresar 0 o dejarlo vac\'io indicar\'a que este par\'ametro
ser\'a ignorado}
\item[Longitud m\'inima:]Es un campo para insertar n\'umeros, este tiene la
funci\'on de indicar que el campo que est\'a siendo creado, cuando se agregue a un proceso siempre debe tener informaci\'on de una longitud
m\'inima del valor introducido.
\footnotemark[\value{footnote}]
\end{description}

Cuando termine de ingresar toda la informaci\'on despliegue el men\'u \blackberry y
seleccione \emph{Guardar}.
\section{Nueva categor\'ia}
\label{sec:nuevaCategoria}
Dir\'ijase a la pantalla inicial, despliegue el men\'u \blackberry y seleccione
\emph{Nuevo}, en este punto se le muestra un listado de los elementos posibles y
all\'i elija la opci\'on \emph{Categor\'ia}.

Se presentar\'a una pantalla en la que podr\'a introducir la descripci\'on de la
nueva categor\'ia.
\footnote{El campo Nombre se considera obligatorio},
cuando termine de ingresarla despliegue el men\'u \blackberry y seleccione
\emph{Guardar}.
\section{Nueva plantilla}
\label{sec:nuevaPlantilla}
Dir\'ijase a la pantalla inicial, despliegue el men\'u \blackberry y seleccione
\emph{Nuevo}, en este punto se le muestra un listado de los elementos posibles y
all\'i elija la opci\'on \emph{Plantilla}.

Se presentar\'a una pantalla en la que podr\'a introducir toda la informaci\'on
necesaria para su nueva plantilla. A continuaci\'on se describen las
caracter\'isticas de cada campo:
\begin{description}
\item[Nombre:]Es un campo de texto usado para diferenciar la plantilla, por lo
tanto se considera obligatorio ingresarlo.
\item[Demandante:]Es una etiqueta no editable que muestra el demandante
seleccionado, para cambiarlo solamente haga click sobre este
\footnote{Tambi\'en puede hacerlo desplegando el men\'u \blackberry y
seleccionando \emph{cambiar}}
e inmediatamente se
le mostrar\'a un listado con los demandantes existentes del cual podr\'a
seleccionar alguno.
\item[Demandado:]Es una etiqueta no editable que muestra el demandado
seleccionado, para cambiarlo solamente haga click sobre este
\footnotemark[\value{footnote}]
e inmediatamente.
se le mostrar\'a un listado con los demandados existentes del cual podr\'a
seleccionar alguno
\item[Juzgado:]Es una etiqueta no editable que muestra el juzgado
seleccionado, para cambiarlo solamente haga click sobre este
\footnotemark[\value{footnote}]
e inmediatamente
se le mostrar\'a un listado con los juzgados existentes del cual podr\'a
seleccionar alguno.
\item[Radicado:]Es un campo de texto en el que usted puede ingresar de manera
opcional el radicado que contendr\'a la plantilla.
\item[Radicado \'unico:]Es un campo de texto en el que usted puede ingresar de
manera opcional el radicado \'unico que contendr\'a la plantilla.
\item[Tipo:]Es un campo de texto en el que usted puede ingresar de manera
opcional el tipo que contendr\'a la plantilla.
\item[Estado:]Es un campo de texto en el que usted puede ingresar de manera
opcional el estado que contendr\'a la plantilla.
\item[Categor\'ia:]Es una etiqueta no editable que muestra la categor\'ia
a la que pertenece la plantilla actualmente, para cambiarla solamente haga click
sobre esta
\footnotemark[\value{footnote}]
e inmediatamente
se le mostrar\'a un listado con las categor\'ias existentes del cual podr\'a
seleccionar alguna.
\item[Prioridad:]Es un campo de selecci\'on, con este usted podr\'a asignarle
la prioridad a su nueva plantilla. \'Uselo haciendo click sobre este y
despu\'es seleccionando el valor deseado.
\item[Notas:]Es un campo de texto en el que usted puede ingresar de manera
opcional la nota que contendr\'a la plantilla.
\end{description}

\subsection{Agregando campos personalizados}
\label{sec:agregarCamposPlantilla}
Despliegue el men\'u \blackberry y seleccione \emph{Agregar campo personalizado}, se
presentar\'a un listado con los campos existentes, all\'i elija el campo
personalizado que desea a\~nadir al proceso.

\subsection{Modificando campos personalizados}
\label{sec:modificarCamposPlantilla}
Sit\'ue el cursor sobre el campo personalizado que desea modificar, y despliegue el
men\'u \blackberry, despu\'es seleccione \emph{Modificar},
se presentar\'a una pantalla en la que podr\'a modificar cada caracter\'istica
del campo personalizado. Esta pantalla se profundiza en la secci\'on
\ref{sec:verCampo} (Pag.\pageref{sec:verCampo}).

\subsection{Eliminando campos personalizado}
\label{sec:eliminarCamposPlantilla}
Sit\'ue el cursor sobre el campo personalizado que desea eliminar, y despliegue el
men\'u \blackberry, despu\'es seleccione \emph{Eliminar}.

Cuando termine de ingresar toda la informaci\'on despliegue el men\'u \blackberry y
seleccione \emph{Guardar}.
\section{Nueva actuaci\'on}
\label{sec:nuevaActuacion}
Dir\'ijase a la pantalla inicial, despliegue el men\'u \blackberry y seleccione
\emph{Nuevo}, en este punto se le muestra un listado de los elementos posibles y
all\'i elija la opci\'on \emph{Actuaci\'on}.

Se presentar\'a una pantalla en la que podr\'a introducir toda la informaci\'on
que puede contener una actuaci\'on. A continuaci\'on se describen las
caracter\'isticas de cada campo:

\begin{description}
\item[Juzgado:]Es una etiqueta no editable que muestra el juzgado
seleccionado, para cambiarlo solamente haga click sobre este
\footnote{Tambi\'en puede hacerlo desplegando el men\'u \blackberry y
seleccionando \emph{cambiar}}
e inmediatamente se le mostrar\'a un listado con los juzgados existentes del
cual podr\'a seleccionar alguno. Este campo se considera obligatorio.
\item[Fecha y fecha pr\'oxima:]Son campos de selecci\'on de fecha, para
modificarlos haga click sobre el elemento que desea editar (la fecha o la hora).
\item[Descripci\'on:]Es un campo de texto en el cual usted describir\'a la
nueva actuaci\'on. Este campo se considera obligatorio.
\end{description}

\subsection{Creando una cita}
\label{sec:crearCita}
Las actuaciones pueden contener citas, las cuales son agregadas al calendario
de su dispositivo \blackberry, estas tienen la posibilidad de tener una alarma
con la la anticipaci\'on que usted desee.

Despliegue el men\'u \blackberry y seleccione \emph{Agregar cita}, se mostrar\'a una
pantalla en la que podr\'a modificar la descripci\'on o la fecha pr\'oxima,
si selecciona el campo \emph{Alarma}, se activar\'a un campo adicional en el
que podr\'a especificar la anticipaci\'on y en el siguiente campo la duraci\'on
en minutos, horas o d\'ias. Cuando finalice presione \emph{Aceptar}
\footnote{La cita solamente ser\'a agregada o modificada en el calendario cuando
usted guarde la actuaci\'on}.

\subsection{Modificando una cita}
\label{sec:modificarCita}
Si la actuaci\'on ya contiene una cita pero desea modificarla, desplegando el
men\'u \blackberry encontrar\'a la opci\'on \emph{Modificar cita}, se
mostrar\'a una pantalla similar a la usada en la creaci\'on de una cita (Ver.
\ref{sec:crearCita}) en esta podr\'a modificar la informaci\'on de la cita
y seleccionando aceptar confirmar\'a los cambios
\footnotemark[\value{footnote}].

\subsection{Eliminando una cita}
\label{sec:eliminarCita}
Si la actuaci\'on ya contiene una cita pero desea eliminarla, desplegando el
men\'u \blackberry encontrar\'a la opci\'on \emph{Eliminar cita}, se mostrar\'a
un di\'alogo de confirmaci\'on y la cita ser\'a eliminada
\footnotemark[\value{footnote}].
\section{Nuevo Proceso}
\label{sec:nuevoProceso}
Dir\'ijase a la pantalla inicial, despliegue el men\'u \blackberry y seleccione
\emph{Nuevo}, en este punto se le muestra un listado de los elementos posibles y
all\'i elija la opci\'on \emph{Proceso}.

Se presentar\'a una pantalla en la que podr\'a introducir toda la informaci\'on
que puede contener un proceso A continuaci\'on se describen las
caracter\'isticas de cada campo:
\begin{description}
\item[Demandante:]Es una etiqueta no editable que muestra el demandante
seleccionado, para cambiarlo solamente haga click sobre este
\footnote{Tambi\'en puede hacerlo desplegando el men\'u \blackberry y
seleccionando \emph{cambiar}}
e inmediatamente se
le mostrar\'a un listado con los demandantes existentes del cual podr\'a
seleccionar alguno.
\item[Demandado:]Es una etiqueta no editable que muestra el demandado
seleccionado, para cambiarlo solamente haga click sobre este
\footnotemark[\value{footnote}]
e inmediatamente
se le mostrar\'a un listado con los demandados existentes del cual podr\'a
seleccionar alguno.
\item[Juzgado:]Es una etiqueta no editable que muestra el juzgado
seleccionado, para cambiarlo solamente haga click sobre este
\footnotemark[\value{footnote}]
e inmediatamente
se le mostrar\'a un listado con los juzgados existentes del cual podr\'a
seleccionar alguno.
\item[Radicado:]Es un campo de texto en el que usted puede ingresar de manera
opcional el radicado que contendr\'a el proceso.
\item[Radicado \'unico:]Es un campo de texto en el que usted puede ingresar de
manera opcional el radicado \'unico que contendr\'a el proceso.
\item[Tipo:]Es un campo de texto en el que usted puede ingresar de manera
opcional el tipo que contendr\'a el proceso.
\item[Estado:]Es un campo de texto en el que usted puede ingresar de manera
opcional el estado que contendr\'a el proceso.
\item[Categor\'ia:]Es una etiqueta no editable que muestra la categor\'ia
a la que pertenece el proceso actualmente, para cambiarla solamente haga click
sobre esta
\footnotemark[\value{footnote}]
e inmediatamente
se le mostrar\'a un listado con las categor\'ias existentes del cual podr\'a
seleccionar alguna.
\item[Prioridad:]Es un campo de selecci\'on, con este usted podr\'a asignarle
la prioridad a su nuevo proceso. \'Uselo haciendo click sobre este y
despu\'es seleccionando el valor deseado.
\item[Notas:]Es un campo de texto en el que usted puede ingresar de manera
opcional la nota que contendr\'a el proceso.
\end{description}

\subsection{Agregando campos personalizados}
\label{sec:agregarCamposProceso}
Despliegue el men\'u \blackberry y seleccione \emph{Agregar campo personalizado}, se
presentar\'a un listado con los campos existentes, all\'i elija el campo
personalizado que desea a\~nadir al proceso.

\subsection{Modificando campos personalizados}
\label{sec:modificarCamposProceso}
Sit\'ue el cursor sobre el campo personalizado que desea modificar, y despliegue el
men\'u \blackberry, despu\'es seleccione \emph{Modificar},
se presentar\'a una pantalla en la que podr\'a modificar cada caracter\'istica
del campo personalizado. Esta pantalla se profundiza en la secci\'on
\ref{sec:verCampo} (Pag.\pageref{sec:verCampo}).

\subsection{Eliminando campos personalizado}
\label{sec:eliminarCamposProceso}
Sit\'ue el cursor sobre el campo personalizado que desea eliminar, y despliegue el
men\'u \blackberry, despu\'es seleccione \emph{Eliminar}.

\subsection{Agregando actuaciones}
\label{sec:agregarActuacionesProceso}
Despliegue el men\'u \blackberry y seleccione \emph{Nueva actuaci\'on} y se
presentar\'a la pantalla para crear una nueva actuaci\'on (Ver.
\ref{sec:nuevaActuacion}).

\subsection{Ver actuaciones}
\label{sec:verActuacionesProceso}
Despliegue el men\'u \blackberry y seleccione \emph{Ver actuaciones}, se le
presentar\'a un listado con las actuaciones pertenecientes a este proceso y
haciendo click en cada una podr\'a ver la informaci\'on que contienen. (Ver.
\ref{sec:listadoActuaciones} Pag. \pageref{sec:listadoActuaciones})

\subsection{Eliminando actuaciones}
\label{sec:eliminarActuacionesProceso}
Despliegue el men\'u \blackberry y seleccione \emph{Ver actuaciones}, se le
presentar\'a un listado con las actuaciones pertenecientes a este proceso y
desplegando el men\'u \blackberry nuevamente selecciona la opci\'on
\emph{Eliminar}.

Cuando termine de ingresar toda la informaci\'on despliegue el men\'u \blackberry y
seleccione \emph{Guardar}.


\section{Proceso a partir de plantilla}
\label{sec:procesoPlantilla}

Dir\'ijase a la pantalla inicial, despliegue el men\'u \blackberry y seleccione
\emph{Nuevo}, en este punto se le muestra un listado de los elementos posibles y
all\'i elija la opci\'on \emph{Proceso a partir de plantilla}.

Se presentar\'a una pantalla con el listado de plantillas existentes y haciendo
click en una plantilla se desplegar\'a el dialogo de crear proceso (Ver.
\ref{sec:nuevoProceso}) con la informaci\'on de la plantilla ya incluída.


