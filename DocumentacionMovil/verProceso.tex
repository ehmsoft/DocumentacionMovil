\section{Ver proceso}
\label{sec:verProceso}
Esta pantalla sirve para ver todo el contenido de un proceso as\'i como sus
actuaciones asociadas, adem\'as podr\'a editar los elementos que desee. Para
acceder a esta despliegue el \emph{Listado de juzgados} (Ver
\ref{sec:listadoProcesos}) y haga click en el proceso que desea ver.

A continuaci\'on se describen las caracter\'isticas de cada campo:

\begin{description}
\item[Demandante:]Es una etiqueta no editable que muestra el demandante
seleccionado, para cambiarlo solamente haga click sobre este
\footnote{Tambi\'en puede hacerlo desplegando el men\'u \blackberry y
seleccionando \emph{cambiar}}
e inmediatamente se
le mostrar\'a un listado con los demandantes existentes del cual podr\'a
seleccionar alguno.
\item[Demandado:]Es una etiqueta no editable que muestra el demandado
seleccionado, para cambiarlo solamente haga click sobre este
\footnotemark[\value{footnote}]
e inmediatamente
se le mostrar\'a un listado con los demandados existentes del cual podr\'a
seleccionar alguno.
\item[Juzgado:]Es una etiqueta no editable que muestra el juzgado
seleccionado, para cambiarlo solamente haga click sobre este
\footnotemark[\value{footnote}]
e inmediatamente
se le mostrar\'a un listado con los juzgados existentes del cual podr\'a
seleccionar alguno.
\item[Radicado:]Es un campo de texto el cual para ser editado debe desplegar el
men\'u \blackberry y seleccionar \emph{Editar}.
\item[Radicado \'unico:]Es un campo de texto el cual para ser editado debe
desplegar el men\'u \blackberry y seleccionar \emph{Editar}.
\item[Tipo:]Es un campo de texto el cual para ser editado debe desplegar el
men\'u \blackberry y seleccionar \emph{Editar}.
\item[Estado:]Es un campo de texto el cual para ser editado debe desplegar el
men\'u \blackberry y seleccionar \emph{Editar}.
\item[Categor\'ia:]Es una etiqueta no editable que muestra la categor\'ia
a la que pertenece el proceso actualmente, para cambiarla solamente haga click
sobre esta
\footnotemark[\value{footnote}]
e inmediatamente se le mostrar\'a un listado con las categor\'ias existentes de
la cual podr\'a seleccionar alguna.
\item[Prioridad:]Es un campo de seleccionar el cual para ser editado debe desplegar
el men\'u \blackberry y seleccionar \emph{Editar}.
\item[Notas:]Es un campo de texto el cual para ser editado debe desplegar el
men\'u \blackberry y seleccionar \emph{Editar}.
\end{description}

\guardarVer

\subsection{Agregando campos personalizados}
\label{sec:agregarCamposProceso}
Despliegue el men\'u \blackberry y seleccione \emph{Agregar campo personalizado}, se
presentar\'a un listado con los campos existentes, all\'i elija el campo
personalizado que desea a\~nadir al proceso.

\subsection{Modificando campos personalizados}
\label{sec:modificarCamposProceso}
Sit\'ue el cursor sobre el campo personalizado que desea modificar, y despliegue el
men\'u \blackberry, despu\'es seleccione \emph{Modificar},
se presentar\'a una pantalla en la que podr\'a modificar cada caracter\'istica
del campo personalizado. Esta pantalla se profundiza en la secci\'on
\ref{sec:verCampo} (Pag.\pageref{sec:verCampo}).

\subsection{Eliminando campos personalizado}
\label{sec:eliminarCamposProceso}
Sit\'ue el cursor sobre el campo personalizado que desea eliminar, y despliegue el
men\'u \blackberry, despu\'es seleccione \emph{Eliminar}.

\subsection{Agregando actuaciones}
\label{sec:agregarActuacionesProceso}
Despliegue el men\'u \blackberry y seleccione \emph{Nueva actuaci\'on} y se
presentar\'a la pantalla para crear una nueva actuaci\'on (Ver.
\ref{sec:nuevaActuacion}).

\subsection{Ver actuaciones}
\label{sec:verActuacionesProceso}
Despliegue el men\'u \blackberry y seleccione \emph{Ver actuaciones}, se le
presentar\'a un listado con las actuaciones pertenecientes a este proceso y
haciendo click en cada una podr\'a ver la informaci\'on que contienen. (Ver.
\ref{sec:listadoActuaciones} Pag. \pageref{sec:listadoActuaciones})

\subsection{Eliminando actuaciones}
\label{sec:eliminarActuacionesProceso}
Despliegue el men\'u \blackberry y seleccione \emph{Ver actuaciones}, se le
presentar\'a un listado con las actuaciones pertenecientes a este proceso y
desplegando el men\'u \blackberry nuevamente selecciona la opci\'on
\emph{Eliminar}.