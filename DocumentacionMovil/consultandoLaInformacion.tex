\chapter{Consultando la informaci\'on}
\label{sec:consultandoLaInformacion}
La consulta de la informaci\'on es posible realizarla por medio de dos
m\'etodos relacionados:

\begin{description}
  \item[M\'etodo 1:]Por medio de los \emph{listados} en los que se puede ver la
  colecci\'on de elementos del mismo tipo.
  Puede acceder a estas por medio del men\'u \blackberry en
  la pantalla inicial, all\'i se selecciona \textit{Listado} (Fig.
\ref{fig:listadoPantallaInicial}) y en la ventaja emergente elije el tipo de
  elemento que se  desea listar (Fig. \ref{fig:listadoListados})
\insertImage{PantallaInicial/MenuPrincipal-Listado}{Men\'u \emph{Listado} en la
pantalla inicial}{listadoPantallaInicial}
\insertImage{Varios/ListaListado}{Listados de elementos}{listadoListados}
  \item[M\'etodo 2:]Por medio de las \emph{Pantallas de edici\'on} o de
  visualizaci\'on, estas sirven para explorar cada elemento individual.
  Puede acceder a estas estando en un listado haciendo click en cualquier
  elemento  se lanzar\'a la pantalla de \emph{ver (editar)} (Fig.
\ref{fig:verProceso})
\insertImage{Ver/Proceso/Normal}{Pantalla de \emph{Ver}
proceso}{verProceso}

\end{description}

A continuaci\'on se describe el procedimiento para cada elemento usado por la
aplicaci\'on.

\section{Listado de actuaciones}
\label{sec:listadoActuaciones}
Dir\'ijase a la pantalla inicial, despliegue el men\'u \blackberry y seleccione
\emph{Listado}, en este punto se le muestra una lista de los elementos posibles
y all\'i elija la opci\'on \emph{Actuaci\'on}, se le presentar\'an sus procesos
de los cuales al seleccionar alguno se le mostrar\'a una pantalla
con sus actuaciones.

Estando en la pantalla podr\'a ver, eliminar y crear nuevas
actuaciones de la siguiente forma:

\begin{description}
\item[Ver y modificar actuaci\'on:]Haciendo click o desplegando el men\'u
\blackberry y seleccionando \emph{Ver}, se le presentar\'a la pantalla de
edici\'on que puede ver en la secci\'on \ref{sec:verActuacion} (P\'ag.
\pageref{sec:verActuacion}).
\item[Eliminar actuaci\'on:]Despliegue el men\'u \blackberry y seleccione
\emph{Eliminar}, confirme su acci\'on y la actuaci\'on ser\'a eliminada
definitivamente.
\item[Crear actuaci\'on:]Despliegue el men\'u \blackberry y seleccione \emph{Nuevo},
se le presentar\'a la pantalla vista en la secci\'on \ref{sec:nuevaActuacion}
(P\'ag. \pageref{sec:nuevaActuacion}).
\footnote{Tambi\'en como primer elemento de la lista se presenta una opci\'on
para crear un nuevo elemento}
\end{description}


\section{Ver actuaci\'on}
\label{sec:verActuacion}
Esta pantalla sirve para ver todo el contenido de una actuaci\'on as\'i como
su cita asociada (en caso de contener alguna). Adem\'as podr\'a editar los
elementos que desee, incluyendo la cita. Para acceder a esta despliegue el
\emph{Listado de actuaciones} (Ver \ref{sec:listadoActuaciones}) y haga click
en la actuaci\'on que desea ver.

Estando en esta pantalla podr\'a visualizar o editar la informaci\'on as\'i:

\begin{description}
\item[Editar campos individuales:]Para editar cualquier campo que contenga
informaci\'on despliegue el men\'u \blackberry y seleccione \emph{Editar},
inmediatamente el campo
cambiar\'a de estado podr\'a y ser modificado.
\item[Editar todo:]despliegue el men\'u \blackberry y seleccione \emph{Editar todo},
de esta forma todos los campos pasar\'an a ser editables.
\item[Eliminar actuaci\'on:]Para eliminar la actuaci\'on que est\'a viendo
actualmente despliegue el men\'u \blackberry y seleccione \emph{Eliminar}, al
confirmar ser\'a eliminada definitivamente.
\end{description}

\guardarVer

\subsection{Creando una cita}
\label{sec:crearCita}
Las actuaciones pueden contener citas, las cuales son agregadas al calendario
de su dispositivo \blackberry, estas tienen la posibilidad de tener una alarma
con la la anticipaci\'on que usted desee.

Despliegue el men\'u \blackberry y seleccione \emph{Agregar cita}, se mostrar\'a una
pantalla en la que podr\'a modificar la descripci\'on o la fecha pr\'oxima,
si selecciona el campo \emph{Alarma}, se activar\'a un campo adicional en el
que podr\'a especificar la anticipaci\'on y en el siguiente campo la duraci\'on
en Minutos, hora o d\'ias. Cuando finalice presione \emph{Aceptar}
\footnote{La cita solamente ser\'a agregada o modificada en el calendario cuando
usted guarde la actuaci\'on}.

\subsection{Modificando una cita}
\label{sec:modificarCita}
Si la actuaci\'on ya contiene una cita pero desea modificarla, desplegando el
men\'u \blackberry encontrar\'a la opci\'on \emph{Modificar cita}, se
mostrar\'a una pantalla similar a la usada en la creaci\'on de una cita (Ver.
\ref{sec:crearCita}) en esta podr\'a modificar la informaci\'on de la cita
y seleccionando aceptar confirmar\'a los cambios
\footnotemark[\value{footnote}].

\subsection{Eliminando una cita}
\label{sec:eliminarCita}
Si la actuaci\'on ya contiene una cita pero desea eliminarla, desplegando el
men\'u \blackberry encontrar\'a la opci\'on \emph{Eliminar cita}, se mostrar\'a
un di\'alogo de confirmaci\'on y la cita ser\'a eliminada
\footnotemark[\value{footnote}].
\section{Listado de categor\'ias}
\label{sec:listadoCategorias}
Dir\'ijase a la pantalla inicial, despliegue el men\'u \blackberry y seleccione
\emph{Listado}, en este punto se le muestra una lista de los elementos posibles
y all\'i elija la opci\'on \emph{Categor\'ia}, se le mostrar\'a una pantalla
con sus categor\'ias.

Estando en la pantalla podr\'a ver, eliminar y crear nuevas
categor\'ias de la siguiente forma:

\begin{description}
\item[Ver y modificar categor\'ia:]Haciendo click o desplegando el men\'u
\blackberry y seleccionando \emph{Ver}, se le presentar\'a la pantalla de
edici\'on que puede ver en la secci\'on \ref{sec:verCategoria} (P\'ag.
\pageref{sec:verCategoria}).
\item[Eliminar categor\'ia:]Despliegue el men\'u \blackberry y seleccione
\emph{Eliminar}, confirme su acci\'on y la categor\'ia ser\'a eliminada
definitivamente.
\item[Crear categor\'ia:]Despliegue el men\'u \blackberry y seleccione \emph{Nuevo},
se le presentar\'a la pantalla vista en la secci\'on \ref{sec:nuevaCategoria}
(P\'ag. \pageref{sec:nuevaCategoria}).
\footnote{Tambi\'en como primer elemento de la lista se presenta una opci\'on
para crear un nuevo elemento}
\end{description}


\section{Ver categor\'ia}
\label{sec:verCategoria}
Esta pantalla sirve para ver y modificar la descripci\'on de la categor\'ia.
Para acceder a esta despliegue el \emph{Listado de categor\'ias}
(Ver \ref{sec:listadoCategorias}) y haga click en la categor\'ia que desea ver.

Estando en esta pantalla podr\'a visualizar o editar la informaci\'on as\'i:

\begin{description}
\item[Editar:]Para editar la descripci\'on despliegue el men\'u \blackberry
y seleccione \emph{Editar}, inmediatamente el campo cambiar\'a de estado
y podr\'a ser modificado.
\item[Eliminar categor\'ia:]Para eliminar la categor\'ia que esta viendo
actualmente despliegue el men\'u \blackberry y seleccione \emph{Eliminar}, al
confirmar la categor\'ia ser\'a eliminada definitivamente.
\end{description}

\guardarVer
\section{Listado de campos personalizados}
\label{sec:listadoCampos}
Dir\'ijase a la pantalla inicial, despliegue el men\'u \blackberry y seleccione
\emph{Listado}, en este punto se le muestra una lista de los elementos posibles
y all\'i elija la opci\'on \emph{Campo personalizado}, se le mostrar\'a una
pantalla con sus campos personalizados.

Estando en la pantalla podr\'a ver, eliminar y crear nuevos campos
personalizados de la siguiente forma:

\begin{description}
\item[Ver y modificar campo personalizado:]Haciendo click o desplegando el men\'u
\blackberry y seleccionando \emph{Ver}, se le presentar\'a la pantalla de
edici\'on que puede ver en la secci\'on \ref{sec:verCampo} (P\'ag.
\pageref{sec:verCampo}).
\item[Eliminar campo personalizado:]Despliegue el men\'u \blackberry y seleccione
\emph{Eliminar}, confirme su acci\'on y el campo personalizado ser\'a eliminado
definitivamente.
\item[Crear campo personalizado:]Despliegue el men\'u \blackberry y seleccione
\emph{Nuevo}, se le presentar\'a la pantalla vista en la
secci\'on \ref{sec:nuevoCampo}
(P\'ag. \pageref{sec:nuevoCampo}).
\footnote{Tambi\'en como primer elemento de la lista se presenta una opci\'on
para crear un nuevo elemento}
\end{description}

\section{Ver campo personalizado}
\label{sec:verCampo}
Esta pantalla sirve para ver todo el contenido de un campo personalizado y
podr\'a editar los elementos que desee. Para acceder a esta
despliegue el \emph{Listado de campos personalizados}
(Ver \ref{sec:listadoCampos}) y haga click en el campo personalizado que desea
ver.

Estando en esta pantalla podr\'a visualizar o editar la informaci\'on as\'i:

\begin{description}
\item[Editar campos individuales:]Para editar cualquier campo que contenga
informaci\'on despliegue el men\'u \blackberry y seleccione \emph{Editar},
inmediatamente el campo
cambiar\'a de estado podr\'a y ser modificado.
\item[Editar todo:]despliegue el men\'u \blackberry y seleccione \emph{Editar todo},
de esta forma todos los campos pasar\'an a ser editables.
\item[Eliminar campo personalizado:]Para eliminar el campo personalizado que
est\'a viendo actualmente despliegue el men\'u \blackberry y
seleccione \emph{Eliminar}, al confirmar, el campo personalizado ser\'a
eliminado definitivamente.
\end{description}
\guardarVer
\section{Listado de demandantes}
\label{sec:listadoDemandantes}
Dir\'ijase a la pantalla inicial, despliegue el men\'u \blackberry y seleccione
\emph{Listado}, en este punto se le muestra una lista de los elementos posibles
y all\'i elija la opci\'on \emph{Demandantes}, se le mostrar\'a una
pantalla con demandantes.

Estando en la pantalla podr\'a ver, eliminar y crear nuevos demandantes de la
siguiente forma:

\begin{description}
\item[Ver y modificar demandante:]Haciendo click o desplegando el men\'u
\blackberry y seleccionando \emph{Ver}, se le presentar\'a la pantalla de
edici\'on que puede ver en la secci\'on \ref{sec:verDemandante} (P\'ag.
\pageref{sec:verDemandante}).
\item[Eliminar demandante:]Despliegue el men\'u \blackberry y seleccione
\emph{Eliminar}, confirme su acci\'on y el demandante ser\'a eliminado
definitivamente.
\item[Crear demandante:]Despliegue el men\'u \blackberry y seleccione
\emph{Nuevo}, se le presentar\'a la pantalla vista en la
secci\'on \ref{sec:nuevoDemandante}
(P\'ag. \pageref{sec:nuevoDemandante}).
\footnote{Tambi\'en como primer elemento de la lista se presenta una opci\'on
para crear un nuevo elemento}
\end{description}

\section{Ver demandante}
\label{sec:verDemandante}
Esta pantalla sirve para ver todo el contenido de un demandante y
podr\'a editar los elementos que desee. Para acceder a esta
despliegue el \emph{Listado de demandantes}
(Ver \ref{sec:listadoDemandantes}) y haga click en el demandante que
desea ver.

Estando en esta pantalla podr\'a visualizar o editar la informaci\'on as\'i:

\begin{description}
\item[Editar campos individuales:]Para editar cualquier campo que contenga
informaci\'on despliegue el men\'u \blackberry y seleccione \emph{Editar},
inmediatamente el campo
cambiar\'a de estado podr\'a y ser modificado.
\item[Editar todo:]despliegue el men\'u \blackberry y seleccione \emph{Editar todo},
de esta forma todos los campos pasar\'an a ser editables.
\item[Eliminar demandante:]Para eliminar el demandante que
est\'a viendo actualmente despliegue el men\'u \blackberry y
seleccione \emph{Eliminar}, al confirmar, el demandante ser\'a
eliminado definitivamente.
\end{description}

\guardarVer
\section{Listado de demandados}
\label{sec:listadoDemandados}
Dir\'ijase a la pantalla inicial, despliegue el men\'u \blackberry y seleccione
\emph{Listado}, en este punto se le muestra una lista de los elementos posibles
y all\'i elija la opci\'on \emph{Demandados}, se le mostrar\'a una
pantalla con demandados.

Estando en la pantalla podr\'a ver, eliminar y crear nuevos demandados de la
siguiente forma:

\begin{description}
\item[Ver y modificar demandado:]Haciendo click o desplegando el men\'u
\blackberry y seleccionando \emph{Ver}, se le presentar\'a la pantalla de
edici\'on que puede ver en la secci\'on \ref{sec:verDemandado} (P\'ag.
\pageref{sec:verDemandado}).
\item[Eliminar demandado:]Despliegue el men\'u \blackberry y seleccione
\emph{Eliminar}, confirme su acci\'on y el demandado ser\'a eliminado
definitivamente.
\item[Crear demandado:]Despliegue el men\'u \blackberry y seleccione
\emph{Nuevo}, se le presentar\'a la pantalla vista en la
secci\'on \ref{sec:nuevoDemandado}
(P\'ag. \pageref{sec:nuevoDemandado}).
\footnote{Tambi\'en como primer elemento de la lista se presenta una opci\'on
para crear un nuevo elemento}
\end{description}

\section{Ver demandado}
\label{sec:verDemandado}
Esta pantalla sirve para ver todo el contenido de un demandado y
podr\'a editar los elementos que desee. Para acceder a esta
despliegue el \emph{Listado de demandados}
(Ver \ref{sec:listadoDemandados}) y haga click en el demandado que
desea ver.

Estando en esta pantalla podr\'a visualizar o editar la informaci\'on as\'i:

\begin{description}
\item[Editar campos individuales:]Para editar cualquier campo que contenga
informaci\'on despliegue el men\'u \blackberry y seleccione \emph{Editar},
inmediatamente el campo
cambiar\'a de estado podr\'a y ser modificado.
\item[Editar todo:]despliegue el men\'u \blackberry y seleccione \emph{Editar todo},
de esta forma todos los campos pasar\'an a ser editables.
\item[Eliminar demandado:]Para eliminar el demandado que
est\'a viendo actualmente despliegue el men\'u \blackberry y
seleccione \emph{Eliminar}, al confirmar, el demandado ser\'a
eliminado definitivamente.
\end{description}

\guardarVer
\section{Listado de juzgados}
\label{sec:listadoJuzgados}
Dir\'ijase a la pantalla inicial, despliegue el men\'u \blackberry y seleccione
\emph{Listado}, en este punto se le muestra una lista de los elementos posibles
y all\'i elija la opci\'on \emph{Juzgados}, se le mostrar\'a una
pantalla con sus juzgados.

Estando en la pantalla podr\'a ver, eliminar y crear nuevos juzgados de la
siguiente forma:

\begin{description}
\item[Ver y modificar juzgado:]Haciendo click o desplegando el men\'u
\blackberry y seleccionando \emph{Ver}, se le presentar\'a la pantalla de
edici\'on que puede ver en la secci\'on \ref{sec:verJuzgado} (P\'ag.
\pageref{sec:verJuzgado}).
\item[Eliminar juzgado:]Despliegue el men\'u \blackberry y seleccione
\emph{Eliminar}, confirme su acci\'on y el juzgado ser\'a eliminado
definitivamente.
\item[Crear juzgado:]Despliegue el men\'u \blackberry y seleccione
\emph{Nuevo}, se le presentar\'a la pantalla vista en la
secci\'on \ref{sec:nuevoJuzgado}
(P\'ag. \pageref{sec:nuevoJuzgado}).
\footnote{Tambi\'en como primer elemento de la lista se presenta una opci\'on
para crear un nuevo elemento}
\end{description}

\section{Ver juzgado}
\label{sec:verJuzgado}
Esta pantalla sirve para ver todo el contenido de un juzgado y
podr\'a editar los elementos que desee. Para acceder a esta
despliegue el \emph{Listado de juzgados}
(Ver \ref{sec:listadoJuzgados}) y haga click en el juzgados que
desea ver.

Estando en esta pantalla podr\'a visualizar o editar la informaci\'on as\'i:

\begin{description}
\item[Editar campos individuales:]Para editar cualquier campo que contenga
informaci\'on despliegue el men\'u \blackberry y seleccione \emph{Editar},
inmediatamente el campo
cambiar\'a de estado podr\'a y ser modificado.
\item[Editar todo:]despliegue el men\'u \blackberry y seleccione \emph{Editar todo},
de esta forma todos los campos pasar\'an a ser editables.
\item[Eliminar juzgado:]Para eliminar el juzgado que
est\'a viendo actualmente despliegue el men\'u \blackberry y
seleccione \emph{Eliminar}, al confirmar ser\'a eliminado definitivamente.
\end{description}

\guardarVer
\section{Listado de procesos}
\label{sec:listadoProcesos}
Dir\'ijase a la pantalla inicial, despliegue el men\'u \blackberry y seleccione
\emph{Listado}, en este punto se le muestra una lista de los elementos posibles
y all\'i elija la opci\'on \emph{Procesos}, se le mostrar\'a una
pantalla con sus procesos.

Estando en la pantalla podr\'a ver, eliminar y crear nuevos procesos de la
siguiente forma:

\begin{description}
\item[Ver y modificar proceso:]Haciendo click o desplegando el men\'u
\blackberry y seleccionando \emph{Ver}, se le presentar\'a la pantalla de
edici\'on que puede ver en la secci\'on \ref{sec:verProceso} (P\'ag.
\pageref{sec:verProceso}).
\item[Eliminar proceso:]Despliegue el men\'u \blackberry y seleccione
\emph{Eliminar}, confirme su acci\'on y el proceso ser\'a eliminado
definitivamente.
\item[Crear proceso:]Despliegue el men\'u \blackberry y seleccione
\emph{Nuevo}, se le presentar\'a la pantalla vista en la
secci\'on \ref{sec:nuevoProceso} (P\'ag. \pageref{sec:nuevoJuzgado}).
\footnote{Tambi\'en como primer elemento de la lista se presenta una opci\'on
para crear un nuevo elemento}
\end{description}

\section{Ver proceso}
\label{sec:verProceso}
Esta pantalla sirve para ver todo el contenido de un proceso as\'i como sus
actuaciones asociadas, adem\'as podr\'a editar los elementos que desee. Para
acceder a esta despliegue el \emph{Listado de juzgados} (Ver
\ref{sec:listadoProcesos}) y haga click en el proceso que desea ver.

A continuaci\'on se describen las caracter\'isticas de cada campo:

\begin{description}
\item[Demandante:]Es una etiqueta no editable que muestra el demandante
seleccionado, para cambiarlo solamente haga click sobre este
\footnote{Tambi\'en puede hacerlo desplegando el men\'u \blackberry y
seleccionando \emph{cambiar}}
e inmediatamente se
le mostrar\'a un listado con los demandantes existentes del cual podr\'a
seleccionar alguno.
\item[Demandado:]Es una etiqueta no editable que muestra el demandado
seleccionado, para cambiarlo solamente haga click sobre este
\footnotemark[\value{footnote}]
e inmediatamente
se le mostrar\'a un listado con los demandados existentes del cual podr\'a
seleccionar alguno.
\item[Juzgado:]Es una etiqueta no editable que muestra el juzgado
seleccionado, para cambiarlo solamente haga click sobre este
\footnotemark[\value{footnote}]
e inmediatamente
se le mostrar\'a un listado con los juzgados existentes del cual podr\'a
seleccionar alguno.
\item[Radicado:]Es un campo de texto el cual para ser editado debe desplegar el
men\'u \blackberry y seleccionar \emph{Editar}.
\item[Radicado \'unico:]Es un campo de texto el cual para ser editado debe
desplegar el men\'u \blackberry y seleccionar \emph{Editar}.
\item[Tipo:]Es un campo de texto el cual para ser editado debe desplegar el
men\'u \blackberry y seleccionar \emph{Editar}.
\item[Estado:]Es un campo de texto el cual para ser editado debe desplegar el
men\'u \blackberry y seleccionar \emph{Editar}.
\item[Categor\'ia:]Es una etiqueta no editable que muestra la categor\'ia
a la que pertenece el proceso actualmente, para cambiarla solamente haga click
sobre esta
\footnotemark[\value{footnote}]
e inmediatamente se le mostrar\'a un listado con las categor\'ias existentes de
la cual podr\'a seleccionar alguna.
\item[Prioridad:]Es un campo de seleccionar el cual para ser editado debe desplegar
el men\'u \blackberry y seleccionar \emph{Editar}.
\item[Notas:]Es un campo de texto el cual para ser editado debe desplegar el
men\'u \blackberry y seleccionar \emph{Editar}.
\end{description}

\guardarVer

\subsection{Agregando campos personalizados}
\label{sec:agregarCamposProceso}
Despliegue el men\'u \blackberry y seleccione \emph{Agregar campo personalizado}, se
presentar\'a un listado con los campos existentes, all\'i elija el campo
personalizado que desea a\~nadir al proceso.

\subsection{Modificando campos personalizados}
\label{sec:modificarCamposProceso}
Sit\'ue el cursor sobre el campo personalizado que desea modificar, y despliegue el
men\'u \blackberry, despu\'es seleccione \emph{Modificar},
se presentar\'a una pantalla en la que podr\'a modificar cada caracter\'istica
del campo personalizado. Esta pantalla se profundiza en la secci\'on
\ref{sec:verCampo} (Pag.\pageref{sec:verCampo}).

\subsection{Eliminando campos personalizado}
\label{sec:eliminarCamposProceso}
Sit\'ue el cursor sobre el campo personalizado que desea eliminar, y despliegue el
men\'u \blackberry, despu\'es seleccione \emph{Eliminar}.

\subsection{Agregando actuaciones}
\label{sec:agregarActuacionesProceso}
Despliegue el men\'u \blackberry y seleccione \emph{Nueva actuaci\'on} y se
presentar\'a la pantalla para crear una nueva actuaci\'on (Ver.
\ref{sec:nuevaActuacion}).

\subsection{Ver actuaciones}
\label{sec:verActuacionesProceso}
Despliegue el men\'u \blackberry y seleccione \emph{Ver actuaciones}, se le
presentar\'a un listado con las actuaciones pertenecientes a este proceso y
haciendo click en cada una podr\'a ver la informaci\'on que contienen. (Ver.
\ref{sec:listadoActuaciones} Pag. \pageref{sec:listadoActuaciones})

\subsection{Eliminando actuaciones}
\label{sec:eliminarActuacionesProceso}
Despliegue el men\'u \blackberry y seleccione \emph{Ver actuaciones}, se le
presentar\'a un listado con las actuaciones pertenecientes a este proceso y
desplegando el men\'u \blackberry nuevamente selecciona la opci\'on
\emph{Eliminar}.
\section{Listado de plantillas}
\label{sec:listadoPlantillas}
Dir\'ijase a la pantalla inicial, despliegue el men\'u \blackberry y seleccione
\emph{Listado}, en este punto se le muestra una lista de los elementos posibles
y all\'i elija la opci\'on \emph{Plantillas}, se le mostrar\'a una
pantalla con sus plantillas.

Estando en la pantalla podr\'a ver, eliminar y crear nuevos plantillas de la
siguiente forma:

\begin{description}
\item[Ver y modificar plantilla:]Haciendo click o desplegando el men\'u
\blackberry y seleccionando \emph{Ver}, se le presentar\'a la pantalla de
edici\'on que puede ver en la secci\'on \ref{sec:verPlantilla} (P\'ag.
\pageref{sec:verPlantilla}).
\item[Eliminar plantilla:]Despliegue el men\'u \blackberry y seleccione
\emph{Eliminar}, confirme su acci\'on y la plantilla ser\'a eliminado
definitivamente.
\item[Crear plantilla:]Despliegue el men\'u \blackberry y seleccione
\emph{Nuevo}, se le presentar\'a la pantalla vista en la
secci\'on \ref{sec:nuevaPlantilla} (P\'ag. \pageref{sec:nuevaPlantilla}).
\footnote{Tambi\'en como primer elemento de la lista se presenta una opci\'on
para crear un nuevo elemento}
\end{description}
\section{Ver plantilla}
\label{sec:verPlantilla}
Esta pantalla sirve para ver todo el contenido de un plantilla y podr\'a editar
los elementos que desee. Para acceder a esta despliegue el \emph{Listado de
plantillas} (Ver \ref{sec:listadoPlantillas}) y haga click en la plantilla que
desea ver.

A continuaci\'on se describen las caracter\'isticas de cada campo:

\begin{description}
\item[Nombre:]Es un campo de texto el cual para ser editado debe desplegar el
men\'u \blackberry y seleccionar \emph{Editar}.
\item[Demandante:]Es una etiqueta no editable que muestra el demandante
seleccionado, para cambiarlo solamente haga click sobre este
\footnote{Tambi\'en puede hacerlo desplegando el men\'u \blackberry y
seleccionando \emph{cambiar}}
e inmediatamente se
le mostrar\'a un listado con los demandantes existentes del cual podr\'a
seleccionar alguno.
\item[Demandado:]Es una etiqueta no editable que muestra el demandado
seleccionado, para cambiarlo solamente haga click sobre este
\footnotemark[\value{footnote}]
e inmediatamente
se le mostrar\'a un listado con los demandados existentes del cual podr\'a
seleccionar alguno.
\item[Juzgado:]Es una etiqueta no editable que muestra el juzgado
seleccionado, para cambiarlo solamente haga click sobre este
\footnotemark[\value{footnote}]
e inmediatamente
se le mostrar\'a un listado con los juzgados existentes del cual podr\'a
seleccionar alguno.
\item[Radicado:]Es un campo de texto el cual para ser editado debe desplegar el
men\'u \blackberry y seleccionar \emph{Editar}.
\item[Radicado \'unico:]Es un campo de texto el cual para ser editado debe
desplegar el men\'u \blackberry y seleccionar \emph{Editar}.
\item[Tipo:]Es un campo de texto el cual para ser editado debe desplegar el
men\'u \blackberry y seleccionar \emph{Editar}.
\item[Estado:]Es un campo de texto el cual para ser editado debe desplegar el
men\'u \blackberry y seleccionar \emph{Editar}.
\item[Categor\'ia:]Es una etiqueta no editable que muestra la categor\'ia
a la que pertenece la plantilla actualmente, para cambiarla solamente haga click
sobre esta
\footnotemark[\value{footnote}]
e inmediatamente se le mostrar\'a un listado con las categor\'ias existentes de
la cual podr\'a seleccionar alguna.
\item[Prioridad:]Es un campo de seleccionar el cual para ser editado debe desplegar
el men\'u \blackberry y seleccionar \emph{Editar}.
\item[Notas:]Es un campo de texto el cual para ser editado debe desplegar el
men\'u \blackberry y seleccionar \emph{Editar}.
\end{description}

\subsection{Agregando campos personalizados}
\label{sec:agregarCamposPlantilla}
Despliegue el men\'u \blackberry y seleccione \emph{Agregar campo personalizado}, se
presentar\'a un listado con los campos existentes, all\'i elija el campo
personalizado que desea a\~nadir al plantilla.

\subsection{Modificando campos personalizados}
\label{sec:modificarCamposPlantilla}
Sit\'ue el cursor sobre el campo personalizado que desea modificar, y despliegue el
men\'u \blackberry, despu\'es seleccione \emph{Modificar},
se presentar\'a una pantalla en la que podr\'a modificar cada caracter\'istica
del campo personalizado. Esta pantalla se profundiza en la secci\'on
\ref{sec:verCampo} (Pag.\pageref{sec:verCampo}).

\subsection{Eliminando campos personalizado}
\label{sec:eliminarCamposPlantilla}
Sit\'ue el cursor sobre el campo personalizado que desea eliminar, y despliegue el
men\'u \blackberry, despu\'es seleccione \emph{Eliminar}.

Cuando termine de ingresar toda la informaci\'on despliegue el men\'u \blackberry y
seleccione \emph{Guardar}.