\section{Ver actuaci\'on}
\label{sec:verActuacion}
Esta pantalla sirve para ver todo el contenido de una actuaci\'on as\'i como
su cita asociada (en caso de contener alguna). Adem\'as podr\'a editar los
elementos que desee, incluyendo la cita. Para acceder a esta despliegue el
\emph{Listado de actuaciones} (Ver \ref{sec:listadoActuaciones}) y haga click
en la actuaci\'on que desea ver.

Estando en esta pantalla podr\'a visualizar o editar la informaci\'on as\'i:

\begin{description}
\item[Editar campos individuales:]Para editar cualquier campo que contenga
informaci\'on despliegue el men\'u \blackberry y seleccione \emph{Editar},
inmediatamente el campo
cambiar\'a de estado podr\'a y ser modificado.
\item[Editar todo:]despliegue el men\'u \blackberry y seleccione \emph{Editar todo},
de esta forma todos los campos pasar\'an a ser editables.
\item[Eliminar actuaci\'on:]Para eliminar la actuaci\'on que est\'a viendo
actualmente despliegue el men\'u \blackberry y seleccione \emph{Eliminar}, al
confirmar ser\'a eliminada definitivamente.
\end{description}

\guardarVer

\subsection{Creando una cita}
\label{sec:crearCita}
Las actuaciones pueden contener citas, las cuales son agregadas al calendario
de su dispositivo \blackberry, estas tienen la posibilidad de tener una alarma
con la la anticipaci\'on que usted desee.

Despliegue el men\'u \blackberry y seleccione \emph{Agregar cita}, se mostrar\'a una
pantalla en la que podr\'a modificar la descripci\'on o la fecha pr\'oxima,
si selecciona el campo \emph{Alarma}, se activar\'a un campo adicional en el
que podr\'a especificar la anticipaci\'on y en el siguiente campo la duraci\'on
en Minutos, hora o d\'ias. Cuando finalice presione \emph{Aceptar}
\footnote{La cita solamente ser\'a agregada o modificada en el calendario cuando
usted guarde la actuaci\'on}.

\subsection{Modificando una cita}
\label{sec:modificarCita}
Si la actuaci\'on ya contiene una cita pero desea modificarla, desplegando el
men\'u \blackberry encontrar\'a la opci\'on \emph{Modificar cita}, se
mostrar\'a una pantalla similar a la usada en la creaci\'on de una cita (Ver.
\ref{sec:crearCita}) en esta podr\'a modificar la informaci\'on de la cita
y seleccionando aceptar confirmar\'a los cambios
\footnotemark[\value{footnote}].

\subsection{Eliminando una cita}
\label{sec:eliminarCita}
Si la actuaci\'on ya contiene una cita pero desea eliminarla, desplegando el
men\'u \blackberry encontrar\'a la opci\'on \emph{Eliminar cita}, se mostrar\'a
un di\'alogo de confirmaci\'on y la cita ser\'a eliminada
\footnotemark[\value{footnote}].