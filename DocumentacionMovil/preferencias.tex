\chapter{Preferencias}
\label{sec:preferencias}
Su aplicaci\'on puede ser modificada en ciertos comportamientos para hacer
m\'as c\'omoda su experiencia.

\insertImage{Preferencias/Preferencias}{Pantalla de
preferencias (Primera parte)}{preferencias}

\section{Mostrar b\'usqueda en todas las pantallas}
\label{sec:busquedaPantallas}
Los listados pueden contener o no un campo de b\'usqueda para hacer m\'as
f\'acil su recorrido por los listados, al seleccionar \emph{Mostrar
b\'usqueda en todas las pantallas} estos ser\'an mostrados en todos los
listados, de lo contrario pasar\'an a ser invisibles pero su funci\'on a\'un
podr\'a ser usada(si el campo no est\'a presente puede teclear y el listado
ser\'a filtrado).
\section{Mostrar t\'itulos de cada pantalla}
\label{sec:tituloPantallas}
Cada pantalla posee un t\'itulo que la diferencia, si desea dejar de verlos
desmarque dicha opci\'on.
\section{Recordar \'ultima categor\'ia}
\label{sec:ultimaCategoria}
Los listados de procesos y plantillas tienen en su parte superior un elemento
con el cual puede realizar un filtro por alguna categor\'ia especifica, si
usted marca la opci\'on \emph{Recordar \'ultima categor\'ia} en las
preferencias cada vez que abandone uno de estos listados la \'ultima
categor\'ia seleccionada ser\'a recordada y al siguiente ingreso estar\'a seleccionada.
\section{Cantidad eventos}
\label{sec:actuacionesCriticas}
En la pantalla inicial se muestran los eventos pr\'oximos a vencerse, la
cantidad listada puede ser modificada cambiando el valor del campo
\emph{Cantidad eventos}.
\section{Fuente de la aplicaci\'on}
\label{sec:fuente}
Cambiando los par\'ametros \emph{Tipo de fuente, Tama\~no fuente y Estilo de
fuente} podr\'a modificar la fuente dentro de toda la aplicaci\'on, podr\'a ver
un ejemplo del resultado en el cuadro que se encuentra despu\'es.

\insertImage{Preferencias/Preferencias2}{Pantalla de
preferencias (Segunda parte)}{preferencias2}
\section{Copia de seguridad}
\label{sec:copiaDeSeguridad}
Puede realizar copia de seguridad de su informaci\'on en cualquier momento por
medio del bot\'on \emph{Copia de seguridad}. Se le presentar\'a una pantalla en
la que puede seleccionar la ubicaci\'on del archivo y despu\'es confirmar la
operaci\'on presionando \emph{Guardar}.
\section{Restaurar preferencias}
Si desea regresar al estado original de todas las preferencias puede presionar
el bot\'on \emph{Restaurar preferencias}.
\section{Cambiar llave de activaci\'on}
Si por alguna raz\'on desea cambiar su llave de activaci\'on puede presionar el
bot\'on \emph{Cambiar Llave de activaci\'on} y se le presentar\'a nuevamente el
di\'alogo para introducirla.